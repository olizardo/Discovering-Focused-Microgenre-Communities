\documentclass[a4paper,12pt]{extarticle}
\usepackage[utf8]{inputenc}
\usepackage{geometry}
    \geometry{a4paper}
\title{Response to Memo}

\begin{document}
\maketitle
\section{Response to Editor's Letter}
The editor noted that the manuscript received a mixed response from the reviewers, with R1 being more positive and R2 and R3 more critical. The more critical reviewers were missing a substantive takeaway for the sociology of taste beyond the proposed methods. The editor's main recommendation was to ``to connect your findings more directly to Poetics (among other) papers that focus on `intra-genre' variation\ldots and culturally dissonant or `discrepant tastes' in music ---ideas often generated through discourse-based or ethnographic approaches that seem to be largely validated by your approach and findings.'' I decided to follow this last route. The last version of the paper trudged through multiple macro genres and their micro-variations in a somewhat listless way. For this version of the paper, I decided to focus on two macrogenres---Metal and Salsa---for which I could find some pointers in the scholarly literature regarding microgenre variation and then show how the link clustering approach can uncover some of the predicted variations in survey data. I think this more focused approach works better and is more effective in showcasing the validity of the proposed approach. I hope the reviewers agree.

\section{Response to R1}
This reviewer found the previous version of the manuscript to be ``incredibly exciting and smart'' and admired the ``relational thinking'' at the basis of the proposed approach. The reviewer also appreciated the broader implications of the argument and approach for social science data analysis more generally.

\subsection{Suggestions I took up}
\begin{itemize}
    \item R1 Suggested I acknowledge precursors of the proposed approach, such as Goldberg's Relational Class Analysis (RCA). I added a relevant discussion of this and other precursors in section 3.1.  
    \item R1 sagaciously pointed to the formal similarity between splitting a macrogenre into microgenre variation and the principle of ``equifinality'' and multiple ``causal recipes'' from Qualitative Comparative Analysis. I address this issue in footnote 3.  
    \item R1 also smartly suggested that I clarify the distinction between the micro-variations uncovered by the link-clustering approach and the more qualitative ones people may make using folk (or even official) classifications. I now address these issues in the concluding section. 
\end{itemize}

\subsection{Suggestions I didn't take up}
\begin{itemize}
    \item R1 suggested that I may perhaps expand on the more ``polemical'' strands of the previous version of the paper, comparing traditional methods against those that exploit the relationality inherent in most data. Since I have excised most of this discussion in this version of the paper, and since going in this direction would have taken the paper too far afield, I decided not to go in this direction. I retained some hints that most methods are affected by the microgenre critique and that adopting the techniques proposed in this paper can help across the board. 
\end{itemize}

\section{Response to R2}
R2 characterized the paper as offering ``an innovative approach to analyzing existing survey datasets on cultural tastes,'' noting that the strength of the paper lied primarily in its ``methodological contribution,'' but that ``the theoretical context and implications of this contribution could be better specified.''

\subsection{Suggestions I took up}
\begin{itemize}
    \item R2 noted that the introduction ended abruptly and that led the reader to lack context as to what the substance and main implications of the analysis that was to follow. To address this issue, I have revised the introduction to provide more context as to the substance of the approach and added a transition paragraph into the methods that provides a synoptic summary of what is to follow. 
    \item R2 suggests that I ``better motivate'' the microgenre concept. I have done this in two ways. First, the current version of the introduction develops the idea in more detail and reviews previous work that points to the concept in more substantive contexts. Second, the idea is also developed in more detail when reviewing previous work focused on the two genre case studies (sections 5.1 and 5.2), which also deploy or hint at the concept in a more substantive context. 
    \item R2 wondered how the microgenres induced from the link-clustering approach and whether the ``micro'' label even fits (e.g., why not meso), which are similar to the wonderings R1 had. As noted, I now address the issues of correspondence (maybe) and scale (relative) in the discussion section.
    \item R2 encouraged me to spell out the implications of the analysis, which they found ``rather narrow'' in the last version, particularly concerning ``suggested directions forward for key debates.'' I think that the macrogenre case studies do that effectively, and I have revised the discussion to reflect that.  
\end{itemize}

\subsection{Suggestions I didn't take up}
\begin{itemize}
    \item R2 complained about the ``lengthy'' methodological setup and suggested that I do more to outline the contribution ``up front'' as a way to obviate the need for a toy example. I tried versions of this, but it didn't work for me. I also like the toy example and think it is necessary for people without a technical background to get behind the black box of what the link-clustering is doing. So I kept it (although I removed the discussion of other methods, as noted so this section is less lengthy).  
\end{itemize}

\section{Response to R3}
R3 thought that the previous version of the paper did a ``fine job of completing what it sets out to do'' but didn't think that the paper made much of a theoretical or empirical contribution, thinking it read more like an introductory methods piece, suitable for pedagogy in a methods class. R3, like R2, was missing a more substantive takeaway that would address ``ongoing debates.'' The reviewer also complained about silly language.

I couldn't do much about the silly language (sorry!). Still, as noted, I rewrote the methods section focusing on two macrogenres selected because either debate existed in the literature or previous work suggested the existence of microgenre variations. I think the presented results address these debates head-on and show how the proposed method can shed light on them and move the discussion forward, so hopefully I did a bit to address this part of the reviewer's critique. 
\end{document}