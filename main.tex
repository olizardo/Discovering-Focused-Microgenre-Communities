\documentclass[a4paper,12pt]{extarticle}
\usepackage[utf8]{inputenc}
\usepackage{graphicx}
\usepackage{natbib}
\usepackage{amsmath}
\usepackage{caption}
\usepackage{subcaption}
\usepackage{geometry}
\usepackage{tablefootnote}
    \geometry{a4paper}
\bibliographystyle{apalike}


\title{From Macrogenres to Microgenres via Relationality}

\begin{document}
\maketitle
\subsubsection*{Abstract} 
Recent work in the sociology of taste has begun to grapple with the relational properties of traditional survey-based data using techniques inspired by network analysis. Despite productive results from this endeavor, critics rightly question the face and ecological validity of the vague macrogenre labels included in standard arts participation surveys (e.g., Classical, Rock, Rap), which feed into these novel methods. In this paper, I propose a link-clustering approach for discovering focused microgenres from standard survey-based information on cultural tastes, exploiting the underlying relational patterns realized by the indirect connectivity structure of genres (via people) in a two-mode network. The link-clustering approach partially answers two of the challenges of macrogenre critics: The fact that actual genres are overlapping and not crisply bounded and that there is hidden heterogeneity within the broad labels we usually focus on. To showcase the fruitfulness of the proposed approach, I engage in two ``case studies'' featuring the vague macro genres of ``Heavy Metal'' and ``Latin/Salsa'' music and show the focused microgenres produced by the link clustering procedure are segmented in ways that help resolve puzzles that have emerged in previous work. 
\newpage

\section{People and Cultural Choices}
Recent work in the sociology of taste has blurred the distinction, foundational for much early work, between ``relational'' network data (codifying the relationships between people) and survey data (codifying the relationship between people and ``variables''). The basic idea is that all data, whether collected purposefully as ``network'' data collected or as part of a traditional social survey, is \textit{relational} data. The main difference is the types of entities related \citep{borgatti_everett97}. Any data source that can be stored in matrix form has ``modes'' (the types of linked entities) and ``ways'' (the number of entities connected by a relation, usually two at a time). The standard network data set is ``one mode'' (we typically look at one type of entity at a time, like people, organizations, schools, countries, and the like) and two ``ways'' (people-by-people or organization-by-organization pairs). In the same way, the usual person-by-variables survey data has two modes and two ways. The relations are not people-by-people but people-by-variables. People are connected to the survey items they answer by ties of affinity, disagreement, choice, or even negatively (by not responding). Following, \citet{breiger74}, it is possible to recover person-by-person (one mode, two ways) matrices from the usual person-by-variables data matrices. People can be connected to others if they share the same values, opinions, tastes, practices, or demographics recorded as variables in the columns of the matrix. 

When transplanted to the sociology of taste, for which survey data has been the workhorse source of insights and empirical generalizations \citep{peterson_kern96, bryson96, vaneijck01,savage_gayo11}, this blurring of the boundaries between the two primary sources of relational data in the social sciences can be revelatory. This goes beyond the fact, to be exploited below, that once we treat survey data as network data, then the entire methodological panoply developed by social network analysts (and increasingly ``network scientists'' working across many disciplines) for the last fifty or so years becomes part of the analytic toolbox. In addition, the entire \textit{conceptual} arsenal of social network theory also becomes available \citep[244]{borgatti_everett97}.

Accordingly, a spate of recent work has begun the job of theoretical translation, enriching the first-generation of work in the sociology of taste with network-flavored concepts. For instance, \citet{pachucki2010cultural} extend Ronald Burt's theory of structural holes for understanding how people can bridge gaps not just in one-mode person-to-person networks but in two-mode person-to-culture networks. Following this lead, \citet{lizardo14} provides a metric for such holes in cultural structure based on Burt's conception of network efficiency that is meant to characterize the slippery concept of ``omnivorousness" concerning genres and cultural forms people engage in. This approach goes beyond summing or counting people's cultural engagements to consider the audience overlap between the genres. Thus, the true omnivore is a person who consumes genres with low audience overlap. Other work in this same vein extends ideas related to positional equivalence in networks \citep[see][]{breiger1976social} to uncover ``blocks'' of respondents that tend to dislike the same set of genres that others like \citep{okada2017structure}. A similar approach, based on positional equivalence in networks, can be extended beyond the study of beliefs and social attitudes to identify respondents who share cultural schemas \citep{goldberg2011mapping}. More recently, \citet{lizardo18} adapts techniques first developed for the study of economic complexity in two-mode networks of geographic sites and products/technologies \citep{hidalgo2009building} for characterizing genres (e.g., popular versus niche) and audiences (e.g., omnivore versus ``univore'') in survey data on cultural tastes. 

\section{The Problem of Genre in the Sociology of Taste}
In this paper, I continue to ride this wave of adapting network approaches to survey data on cultural tastes to tackle a fundamental problem (both substantive and measurement-wise) in the sociology of taste, namely, the problem of \textit{genre}. In a foundational paper, \citet{dimaggio1987classification} provided a prescient conceptualization of the core constructs of the sociology of taste in a way that exploited the network imagery that has become a routine reality of late. In the paper, \citet[244]{dimaggio1987classification} asked us to

\begin{quote}
   {\dots}imagine a matrix defined by persons on the vertical axis and artworks on the horizontal axis, with{\dots}signifying relationships (knowledge about, like for, dislike of) between person and artworks, genres consist of those sets of works which bear similar relations to the same set of persons. The logic behind this imagery will be familiar to students of network analysis as one of ``structural equivalence.''
\end{quote}

Thus, for DiMaggio, genres are dual entities precisely in Breiger's (1974) sense noted earlier. Genre categories are composed of the audiences that engage them (are subsets of the larger set of people). People, on the other hand, are related to one another (e.g., via relations of similarity, opposition, or non-overlap) by the genres they choose, and genres related to one another via overlaps (intersections) in the sets of people who choose them.

Like much network theorizing, this conceptualization is elegant but gets messy in application. As \citet[149]{lena2015relational} has noted, ``[n]o ordering principle is as fundamental to culture as genre'' yet none is also as vexed. Lena raises one fundamental issue: Sociologists have relational intuitions about genres but tend to default to substantive musicological definitions for convenience. Thus, when considering which genres to include in a survey on cultural tastes, the tendency is to pick off-the-shelf from the menu of institutionalized---usually at the level of the culture industry or field of cultural production---genre categories. The problem with these folk genre classifications is that they miss the fuzzy, overlapping ways real-world genres are organized. This leads us to treat what are, in fact, contested boundaries as if they were natural, crisp boundaries (e.g., assigning respondents who like ``Rap'' to a different taste community than those who like ``Opera."). This is the problem of \textit{overlapping boundaries among genre categories}. 

The other issue has to do with the level of aggregation. For convenience’s sake, survey data must settle on a ``middle'' level of categorization so as not to tax respondents' patience (or knowledge). Thus, the standard ``lists'' of cultural genres, such as ``Pop Art,'' ``Romantic Comedy," or ``Blues.'' However, as recent work points out, people, venues, fandoms, scenes, subcultures, communities, organizations, critics, gatekeepers, and even states and other powerful institutions, make sociologically relevant distinctions \textit{within} macrogenres categories \citep{hesmondhalgh2005subcultures, Holt2007, Van_Poecke2018, Hield2014-xe}. Genres may also develop and accumulate variations as they ``travel'' across production, dissemination, and consumption sites in their historical trajectory \citep{Lena2012}. 

For instance, in a study of rap songs that charted in the Top 100 R\&B Billboard charts from 1979-1995, Lena \citeyearpar[299]{lena2004meaning}, used a ``combination of lyrical, vocal, and musical attributes, in combination with the record label and geographic location of the producers'' to code for thirteen distinct ``rap'' microgenres, including ``booty rap, crossover rap, don rap, dirty south rap, east coast gangsta rap, g funk, jazz rap, new jack swing, parody rap, pimp rap, race rap, rock rap, and west coast gangsta rap.'' These microgenres are embedded within larger macrogenres like ``rap and Hip Hop'' in complex ways that the standard survey approach misses.

In the same way, within the Sheffield ``musicworld,'' the broad genre category of ``Folk''---one usually included in the typical survey questionnaire like the one used in this paper---can refer to at least three existing microgenres: A ``traditionalist'' one---see \citep{Lena2012}---composed mainly of late nineteenth and early twentieth English ballads documenting working-class and mining-town life, a ``scene'' genre centered on 1960s political protest and pop songs (\`{a} la Bob Dylan) or more contemporary ``avant-garde'' genre featuring reflexive, sometimes parodic take on the previous two, and centered on blurring performer-audience boundaries. As Hield and Crossley \citeyearpar[197]{Hield2014-xe} note, ``[t]hese and other such distinctions are not registered in\ldots surveys.''

\section{What to do about the problem of genre}
One solution to the problem posed by vague macrogenre labels, suggested by \citet{vlegels2015music, vlegels2017music} concerning musical genres, is to ``change the mode'' and ``drop the label'' by asking respondents not just for their levels of engagement or liking of broad macrogenre labels but for performers within those labels \citep{nault2021social}. This work shows that once these ``microgenre'' distinctions based on performers are made, some core empirical generalizations in the sociology of taste (e.g., highly educated people like classical music) are contradicted or at least heavily qualified. This \textit{microgenre heterogeneity} critique thus asserts that the standard genre labels studied by social scientists hide as much as they reveal because microgenre communities can be internally diverse and even contain mutually opposed groups in the social space \citep{flemmen_etal18}. For people, the unit of selection and judgment is the {\em microgenre}, not the broad genre classification. So, it is unclear what to do (and how to interpret) the usual data collected by sociologists who study taste using surveys. As Jarness \citeyearpar{jarness2015} notes, ``researchers (have had to) deploy crude measures based on wide genre categories, such as pop/rock, classical and country and western in the mapping of musical tastes. In this way, they have tended to obscure that sociologically significant taste distinctions exist within these somewhat arbitrary genre categories.''

But what happens when the data at our disposal (as with long-standing surveys such as the Survey for Public Participation in the Arts or the General Social Survey) is at the macrogenre label? Both overlapping boundary and microgenre heterogeneity critics of traditional genre labels would say to ignore these data and collect new data that does not rely on misleading macrogenre labels. I propose another solution: {\em Exploit the inherent relationality embedded in survey data responses} \citep{goldberg2011mapping, boutyline2017belief, lizardo18}. More specifically, my point will be that by exploiting the intrinsic relationality already contained in the usual survey data (when treated as two-mode network data), old data can reveal insights that pertain to the very features sociological critics of the macrogenre label concept say are missing. In particular, it is possible to use embedded sets of relations between people and genres to discover the heterogeneous microgenre communities critics say are hidden by macrogenre labels. This approach is thus ideal for making new use of old data while sidestepping some, but not all, of the limitations of the microgenre critique. 	

\subsection{Precursors of the Proposed Approach}
The approach proposed here builds on the core ideas of ``duality'' and ``relationality'' that have become foundational for formal approaches to measuring culture in the sociology of taste \citep{mutzel2020duality, mohr2015formal}. In this sense, I followed a path already trailed by others. For instance, Goldberg's \citeyearpar{goldberg2011mapping} {\em Relational Class Analysis} (RCA) ``classifies the people''  included a survey by partitioning a person-by-person matrix built from higher order similarities (sums of differences-of-differences) between each pair of person's corresponding response vector coding their stances (e.g., like, dislike, neutral) toward a set of cultural objects (e.g., genres, attitude items). Boutyline's \citeyearpar{boutyline2017improving} {\em Correlational Class Analysis} (CCA) also classifies the people but this time by using the absolute value of the correlation distance between response vectors to build the person-by-person similarity matrix. Boutyline and Vaisey's {\em Belief Network Analysis} (BNA) classifies the objects people connect to (in their case, beliefs, but could also be genres) by creating a weighted object network, where the weight of the tie between two objects (attitude toward government spending and attitude toward immigration) is given by the magnitude of their correlation coefficient in the survey. 

As we will see, I propose a complementary and synergistic approach with these previous efforts. More to the point, the relational identification of microgenres can improve the applicability of techniques like RCA, CCA, and BNA (and related techniques) because, just like standard statistical approaches (like Factor Analysis, Principal Component Analysis, Latent Class Analysis, and so forth), and even techniques like Multiple Correspondence Analysis sometimes opposed to the traditional statistical ones, these newfangled relational techniques are also dragged down by the macrogenre curse. Thus, RCA and CCA are bound to compute relationality and correlation distances within people but rely on vague macrogenre labels. 

The resulting evaluation logics are thus interpreted using a macrogenre lens---e.g., ``Anything but Heavy Metal'' or ``Highbrow versus Lowbrow" \citep[e.g.,][]{goldberg2011mapping, willekens2022cultural}. But what if what one group means by ``Heavy Metal" differs from what another group means by the same vague label? Are we sure that certain {\em all} types of classical music are ``opposed'' to or divergent from engagement with Country or Hip Hop? Some versions of Indie/Alt Rock are today considered ``highbrow'' as Classical music (used to be) \citep{Van_Poecke2018}. Do high-cultural capital respondents inherently shun every variant of Country? As \citet{lembo2017three} shows, there are microgenres of Country music---in that case, featuring a traditionalist appreciation for the ``hard'' ``Honky Tonk'' sound developed in 1920s Texas---that appeal to highly educated audiences and, thus, to people who are also likely also listen to Musical Theater compositions, Jazz, and Opera. Even though the most likely answers to the preceding rhetorical suggestions fall on the side of significant microgenre variation, quantitative techniques researchers use in the sociology of taste must work with the (broad) genre categories included in the survey. As such, they are powerless to deal with the (reasonable) objections brought up by the microgenre dissenters. 

Importantly, this implies that the clumping of vague macrogenre labels into even more ambiguous meta-macro genres (discourses, styles, logics, schemas) like ``highbrow'' ``Popular," or ``Folk" could be a spurious by-product of the vague labeling (as microgenres critics have long suspected). This matters because such macrogenre clumping does much conceptual (and technical) work in our data reduction, interpretation, and substantive theorizing efforts. The result is that it is likely that people are classified as following the same cultural logic of genre valuation and evaluation when they follow distinct logics. A key implication of the approach to be described below is that regardless of the particular technique used, sociologists of taste are likely to underestimate population heterogeneity in the logics of classification and evaluation people use to navigate the cultural space \citep{lahire2008individual}. This heterogeneity has to be {\em revealed} before constructing belief or attitude networks, classifying people into schematic classes, embedding genres into a dimensional space, or putting people into latent classes.

In the remainder of the paper, I introduce the link-clustering approach to eliciting microgenres using the relational information embedded in the joint choices (and avoidance) of macrogenres available in traditional survey data. I first introduce the motivation behind the method using a simple ``toy example'' and then move to more substantive ground in two ``case studies'' concerning two genres for which substantiated proposals regarding micro-variations exist in the literature. I show that the proposed approach extracts novel insights from survey data that help address, shed light on, and adjudicate these debates at a population level while providing novel leads to future work. 

\section{Analytic Approach}
\subsection{Data to be Used in the Study}
The empirical part of the study uses data on the musical tastes of Americans collected with Sara Skiles in the summer of 2012 \citep{lizardo_skiles15, lizardo_skiles16}. The survey was (partially) designed to replicate the General Social Survey (GSS) 1993 Culture Module. This is now a canonical data set providing the empirical basis for a variety of analyses (and reanalyses) in the sociology of taste \citep[e.g.,][]{bryson96, goldberg2011mapping, schultz2010strength, han2003unraveling, tampubolon2008revisiting, okada2017structure}.  The data were collected by an organization then called Survey Sample International (SSI), a private firm specializing in sampling, data collection, and analysis. SSI managed recruitment and participation invitation tasks to generate a panel of adults from which our working sample was drawn. Survey respondents were selected from the panel for participation based on age, gender, race, education, and geographic region to approximate a sample representative of the U.S. population ($n = 2,263$). 

Like GSS 1993, SSI 2012 included items assessing respondents’ likes and dislikes (as well as a middle category of ``mixed feelings'') for 20 categories of musical style: Classical/Symphony and Chamber, Opera/Operetta, Jazz, Broadway Musicals/Show Tunes, Mood/Easy Listening, Big Band/Swing, Classic Rock/Oldies, Country, Bluegrass, Folk, Hymns/Gospel, Latin/Spanish/Salsa, Rap/Hip Hop, Blues/R\&B, Reggae, Rop 40/Pop, Contemporary Rock, Indie/Alternative Rock, Dance/Club/Electronic, and Hard Rock/Heavy metal. In addition to providing a taste judgment for each item, respondents were also asked if they regularly listened to each. The two-mode network we will be working with is thus of dimensions $2263$ (people) $\times$ $20$ (genres). This defines an affiliation matrix $\mathbf{A}$ of the same row and column dimensions, with entries ($a_{ij}$) equal to 1 if person $i$ likes \textit{and} listens to macrogenre label $j$ and zero otherwise. 

\subsection{The Link Clustering Approach}
I propose that exploiting {\em relationality} hidden in the usual survey data collecting information on people and genres can help us make headway on the microgenre critique outlined in the preceding. As noted at the outset, this requires that we stop looking at the usual survey data that forms the bulk of empirical material in the quantitative study of taste as ``survey data'' (e.g., data collected on the characteristics of individuals in the form of ``variables''). As noted, this assumption is baked into traditional techniques (like Factor Analysis or Latent Class Analysis) that fit statistical models to such data. But it is also embedded in ``relational'' techniques like Multiple Correspondence Analysis which, while not steeped in the preoccupations of standard statistical inference (although they could be), divert our attention to the indirect relations between macrogenres (or between people), usually conceptualized as distances in a social space \citep{flemmen_etal18}. This kind of indirect relationality is fine and dandy. Still, as already noted in the case of RCA, CCA, and BNA, all the relationality built into MCA is powerless if what is fed {\em into} the process is the same old, limited, overextended macrogenre categories. 

\subsubsection{A Toy Example}
I address the microgenre issue using a technique for overlapping community detection called \textit{link clustering} \citep{ahn_etal10}. The basic idea is simple and illustrated in the toy example in Figure~\ref{fig:link-toy1}.\footnote{Code to replicate all of the analyses reported in this paper and software implementing the link clustering procedure, can be found at \nolinkurl{https://github.com/olizardo/Discovering-Focused-Microgenre-Communities}}. Breaking with approaches that attempt to cluster people and genres by focusing on the people or the genres, we take the person-to-genre {\em link} as the unit of analysis and focus our classification energies on those. Classifying the links gives (by definition) a classification of the entities at the end of each link (people and genres) for free. Moreover, because most people choose multiple genres, and all genres ``choose'' multiple people, this classification guarantees that genres get assigned to multiple \textit{overlapping} clusters while splitting up the macrogenres into more relationally focused clumps. This allows us to make headway on both horns of the microgenre critique---namely, overlapping categories and internal differentiation---at once.

\begin{figure}[t!]
    \captionsetup[subfigure]{font=footnotesize,labelfont=footnotesize}
     \centering
     \begin{subfigure}[b]{0.6\textwidth}
        \includegraphics[width=1.0\textwidth]{Toy/link-clust-toy1.png}
        \caption{Two person-genre edges sharing the same genre.}
        \label{fig:link-toy1}
    \end{subfigure} 
     \begin{subfigure}[b]{0.3\textwidth}
        \includegraphics[width=1.0\textwidth]{Toy/link-clust-toy2.png}
        \caption{Similarity between person-genre edges depends on the other genres people are linked to.}
        \label{fig:link-toy2}
    \end{subfigure}
\end{figure}
 
How does this work? As Figure~\ref{fig:link-toy1} shows, once we string out the original two-mode person-by-network in our toy example into an edge list (now containing eleven person-genre edges), where the cases are the person to genre pairs that are marked by a ``1'' in the original matrix, it is possible to ask: ``How similar is one edge to another?'' We can answer that question as follows. The similarity of one person-to-genre edge to another, when both edges point to the same genre \citep{ahn_etal10}, should be proportional to the overlap between the ``cultural ego networks'' of the two people at the tail end of the edges. As \citet{lizardo14} defines it, for any respondent $i$, in a survey on cultural taste, the cultural ego network is simply the set of genres $\{G_1, G_2,\dots G_k\}$ chosen by the person, where $k$ is the total number of genres selected (``omnivorousness by volume''). 

Going back to our running toy example, we are tasked with computing the similarities between $\frac{11 \times (11-1)}{2} = 55$ person-to-genre pairs. Suppose we wanted to calculate the similarity between the $P_1-G_2$ and $P_3-G_2$ person-to-genre edges in the two-mode network. In Figure~\ref{fig:link-toy1}, these two links are highlighted in yellow by the ``strung out'' edgelist and are shown in red in the corresponding bipartite subgraph shown in Figure~\ref{fig:link-toy2}. The usual Jaccard distance then gives the similarity ($S$) between the two links:

\begin{equation}
    S(P_1G_2, P_3G_2) = \frac{n_+(P_1) \cap n_+(P_3)}{n_+(P_1) \cup n_+(P_3)}
\end{equation}

Where $n_+(P_1)$ is the cultural ego network of Person 1 and $n_+(P_3)$ is the cultural ego network of Person 3. The formula says that the similarity between two person-to-genre links sharing a genre, in this case, the similarity between the $P_1-G_2$ edge and the $P_3-G_2$ edge, is given by the intersection of Person 1's and Person 3's cultural ego network divided by their union. When the cultural ego networks of two people are the same $S=1$ when they are entirely disjoint $S = 0$, all other cases of person-to-genre edges sharing the same genre return a number between zero and one ($0 > S < 1$).\footnote{Note that the only substantive similarities we care about are between person-to-genre edges that share the same genre but have different people attached to them at the other end. All other person-to-genre edges similarities featuring people connected to different genres are set to zero.} In the toy example case, $P_1$ and $P_3$ consume two genres out of the four possible ones they could consume in common, resulting in an edge similarity score of $\frac{2}{4} = 0.50$. 

\begin{figure}
    \captionsetup[subfigure]{font=footnotesize,labelfont=footnotesize}
    \centering
     \begin{subfigure}[b]{0.4\textwidth}
        \includegraphics[width=1.0\textwidth]{Toy/toy-sim.png}
        \caption{Edge similarity matrix from toy example.}
        \label{tab:toy-sim}
    \end{subfigure}
     \begin{subfigure}[b]{0.4\textwidth}
        \includegraphics[width=1.0\textwidth]{Toy/toy-dis.png}
        \caption{Edge distance matrix from toy example.}
        \label{tab:toy-dis}
    \end{subfigure}
     \begin{subfigure}[b]{0.6\textwidth}
        \includegraphics[width=1.0\textwidth]{Toy/link-clust-toy.png}
        \caption{Dendrogram from Ward clustering of toy example edge distance matrix cut at four clusters.}
        \label{fig:link-toy}
    \end{subfigure}
    \caption{Toy illustration of how the link clustering approach works.}
    \label{fig:link-toy-ex}
 \end{figure}

If we do that for all pairs of person-to-culture genres sharing one genre node in common, we end up with the two-way, one-mode (with the only mode being the edges) $11 \times 11$ similarity matrix shown in Figure~\ref{tab:toy-sim}. Note two features of the similarity matrix. First, by definition, two person-to-genre links from different genres to the same person are maximally similar (as the overlap is 1.0). This means that only links that go from other people to the same genre exhibit overlap variation, and this is driven (``reflectively'' \citep{lizardo18}) by the overlap between the neighborhoods of the other mode (people). Next, we transform the similarity matrix into a {\em dissimilarity} matrix by subtracting one from each entry (shown in  Figure~\ref{tab:toy-dis}). We then subject this matrix to hierarchical clustering using Ward's \citeyearpar{ward63} method. 

The resulting dendrogram from the cluster analysis is shown in Figure~\ref{fig:link-toy}. Note that moving from the top down, the dendrogram splits person-to-genre links according to the macrogenre label. Macrogenre information is preserved in the clustering and can be recovered by splitting the dendrogram at a height where the number of clusters equals the original number of macrogenres (in this toy case, four). Moving from the bottom up, the hierarchical clustering of the dissimilarity matrix yields {\em link communities} \citep{ahn_etal10}, in which both people nodes, but, most crucially, genres nodes are assigned to different clumps. In the limiting case, each person-to-genre link is assigned to a single cluster, namely, the dendrogram's bottom-most ``leaves.'' 

The more interesting thing is that as we move up to a point below macrogenres clusters and above the leaves, we see a partition of {\em varieties} of macrogenres. For instance, macrogenre $G4$ is split into two focused microgenre varieties. The first is attached to people $P2$ and $P4$, and the second is connected to people $P3$ and $P5$, respectively. The idea is that the macrogenre nodes (e.g., ``Classical'') assigned to different clumps represent micro-variations of the macrogenre label that differ primarily relationally or structurally; different ``types'' of ``Classical'' are different because their audiences make distinct choices {\em with respect} to the other genres in the survey. 

\begin{figure}[t!]
     \begin{subfigure}[b]{0.5\textwidth}
        \centering
        \includegraphics[width=1.0\textwidth]{Plots/Dend/macro.png}
        \caption{c = 8, k = 20}
        \label{fig:dend-macro}
    \end{subfigure} 
     \begin{subfigure}[b]{0.5\textwidth}
        \centering
        \includegraphics[width=1.0\textwidth]{Plots/Dend/macro-lab.png}
        \caption{c = 8, k = 20}
        \label{fig:dend-macro-labels}
    \end{subfigure} 
     \begin{subfigure}[b]{0.5\textwidth}
        \centering
        \includegraphics[width=1.0\textwidth]{Plots/Dend/metal-branches.png}
        \caption{k = 3}
        \label{fig:dend-micro-metal}
    \end{subfigure}
     \begin{subfigure}[b]{0.5\textwidth}
        \centering
        \includegraphics[width=1.0\textwidth]{Plots/Dend/salsa-branches.png}
        \caption{k = 3}
        \label{fig:dend-micro-salsa}
    \end{subfigure}
    \caption{}
    \label{fig:dend}
 \end{figure}
 
The resulting partition has two attractive (and desirable) properties. First, the number of genre communities that \textit{people} belong to is deterministic, and it is given by the number of macrogenre labels they initially chose. Thus, link clustering preserves the cultural ego network degrees (omnivorousness by volume) of the people mode \citep{lizardo14}. Second, the number of microgenres into which the macrogenres are split is {\em not} deterministic. Instead, it is data-driven (discovered or learned) and cannot be pre-specified in advance. It is a function of relational information implicit in the overlap structure of the cultural ego networks of people in the data. Thus, link community detection of the person-to-genre network allows us to go from a situation starting with a person-to-genre network featuring a relatively small number of macrogenres and end up with an enlarged two-mode network with the same number of nodes in one mode (the people) but many more nodes in the other mode (the microgenres). How many microgenres ($N_m$) emerge is up to the analyst as it depends on where we ``cut'' the resulting clustering dendrogram. This will be somewhere between $N_g \geq N_m \leq N_l$, where $N_g$ is the number of original (macro) genres and $N_l$ is the number of person-to-genre links.

\subsection{Discovering Microgenre Communities in Real Data}
Let us see how this process works in real data. Recall that the data features 2,263 people choosing up to 20 macrogenre labels. When strung out as an edgelist, this results in 9,216 person-to-genre connections in the data. The resulting $9216 \times 9,216$ matrix, containing Jaccard similarities among person-to-genre links sharing a node in either of the two modes, is then the input to an agglomerative hierarchical clustering algorithm using Ward's \citeyearpar{ward63} method. 

The hierarchical clustering process proceeds as follows. Initially, each person-to-genre link is assigned to a singleton cluster. Then, in the second time step, the pairs of links with the largest Jaccard similarities (smallest distances) are put in the same clump. This continues at each time step, where pairs of links with the largest similarity are chosen, and their respective communities are merged. This process is repeated until all links belong to a single cluster. The history of the clustering process is then stored in a dendrogram, which encodes all the information on the hierarchical organization of the genre communities. The height of the dendrogram provides information about the strength of the genre communities. As we have seen, the highest levels reproduce the original macrogenres, while the lower levels uncover more focused microgenres embedded within them. This is shown in Figure~\ref{fig:dend}. 

Figure~\ref{fig:dend-macro} shows that when we cut the dendrogram at a high level (e.g., $C = 8$), we reproduce the original twenty macrogenres we began with as the ``discovered'' link communities. As Figure~\ref{fig:dend-macro-labels} shows, the agglomerative link clustering procedure roughly arranges the macrogenre levels by their original popularity (number of person-to-genre links). From left to right, these are Classic Rock/Oldies, Pop/Top 40, Country, Classical, Easy Listening, Contemporary Rock, Blues/R\&B, Rap/Hip Hop, Jazz, /EDM, Dance/Club/EDM Gospel, Indie Alternative, Broadway/Musicals, Heavy Metal/Hard Rock, Latin/Spanish/Salsa, Reggae, Big Band, Folk, Bluegrass, and Opera. 

\subsubsection{Where to cut?}
Data scientists lose sleep over choosing a cut value when performing cluster analysis. One advantage of the link clustering approach is that we always know what we are doing because microgenres are strictly nested within the original macrogenres. Choosing a smaller cut value produces finer-grained microgenres (perhaps at the expense of analytic tractability and interpretability), and selecting a higher value returns us to broader genre communities closer to the original macrogenre labels we began with (with the limiting case as shown in Figure~\ref{fig:dend-macro} being the original macro genre labels themselves). At any cut value $C$, the more popular macrogenres produce more microgenre communities, while the less popular ones produce a smaller number. 

Another approach, and the one I will be using in the case studies that follow, is to pick a macrogenre label (a branch in the larger dendrogram), decide \textit{ex-ante} how many microgenres it makes sense to analyze, and then cut that branch of the dendrogram at a value that returns the desired number of microgenres. That is, fix $k$ and choose a $C$ value that satisfies it. For instance, Figures~\ref{fig:dend-micro-metal} and~\ref{fig:dend-micro-salsa} show the ``Metal'' and ``Salsa'' branches of Figure~\ref{fig:dend-macro} cut at a height that returns three microgenres for each macrogenre label. I chose these two macrogenres as examples because they will be the basis of the following genre case studies. 
 
\section{Two Genre Case Studies}
We have shown that we can perform a link clustering and that this clustering returns a nice set of microgenres nested within the usual macrogenres in the survey (see Figures~\ref{fig:dend-micro-metal} and~\ref{fig:dend-micro-salsa}). But does this exercise result in any substantive gains? In this section, I show that the link clustering approach does allow you to extract new insights from old data. There are many ways we can proceed. I present two ``case studies'' of macrogenre labels and the resulting microgenre variations returned by the link clustering procedure. 

My selection criteria were as follows: I picked two macrogenre labels for which I could find some substantiated claim in the scholarly literature regarding possible microgenre variation. Naturally, general claims about microgenre variation apply to almost all macrogenre labels. Therefore, the ``substantiated in the scholarly literature" serves as a way to narrow the field. My other criterion is for debate or uncertainty in the scholarly literature regarding the existence or overall status of the presumed microgenre variations. Using these criteria, I settled on Heavy Metal and Latin/Spanish Music/Salsa. As we will see, Metal fits both bills, as a lively debate already exists regarding its status as non-elite or an elite cultural form \citep{tampubolon2008revisiting}. Latin/Spanish/Salsa represents a more ``minor'' case study, but one that has also been documented to exhibit the sort of microgenre variation we seek \citep{Bachmayer2014-pk}. As such, Salsa also showcases the usefulness of the link clustering approach for resolving critical issues in the literature and extracting new insights from existing data.

\subsection{Heavy Metal}
Heavy Metal is one of the most storied macrogenre labels in the sociology of taste, as featured in the title of Bethany Bryson's \citeyearpar{bryson96} now classic paper. Since then, it has been the subject of scholarly debate regarding its possibly changing status in the cultural stratification order \citep{tampubolon2008revisiting, goldberg2011mapping, lizardo_skiles15}. Heavy Metal stood out in Bryson's study for two reasons. First, it was the most disliked genre in the 1993 General Social Survey culture module data---one of the most heavily analyzed data sources in the sociology of taste---hence the ``Anything But'' in the title of Bryson's eponymous piece. Second, it was the genre most favored by working-class audiences but hated by individuals with high education. Thus, it served as the linchpin of Bryson's symbolic exclusion argument; omnivores were omnivorous up to a point, except when it came to those genres most favored by those with low education. 

Regardless, the picture we got of Metal concerning audience segmentation was clear: It was a genre that primarily appealed to working-class, low-education white men and was rejected by everyone else. Lizardo and Skiles \citeyearpar[][6, table 2]{lizardo_skiles16} used the same data source as in this paper (data from 2012) and found that, nineteen years later, Metal was still the most disliked of the macrogenre labels in a representative sample of Americans. Not only that but when asked what the typical fan of metal looked like, Americans agreed with the stereotype: Low education, ``lower'' and working-class white men \citep[][7, table 3]{lizardo_skiles16}. 

Yet, cracks began to appear in this picture as soon as it was hung. First, Tampubolon \citeyearpar{tampubolon2008revisiting} re-analyzed the Bryson data using a different methodological approach (latent-class analysis), making other inclusion decisions regarding the ``don't know'' responses. Tampubolon shows that including those ``don't know'' responses significantly changes things. Concerning Metal, it indicates that it is intensely disliked among low-education people, more than would have been surmised in Bryson's original study (because low-education people are more likely to say ``don't know'' to genre questions and are thus more likely to be dropped from the sample when these answers are also excluded), so it is not just a top-down symbolic exclusion story. For Tampubolon, this makes Metal ``quite exceptional'' since it is disliked across the board and even more by people with low education. 

Tampubolon's work indicates that because Metal's audience is composed of people with low education, it does not mean that the rest of the less-educated population loves it; they hate it passionately. Tampubolon's analysis of ``Like'' responses also shows that Metal, rather than being exclusively preferred by a univorous group of ``Metalheads'' (who only like metal and dislike everything else), also makes it into the taste profile of one of two omnivorous groups uncovered in his analysis. Thus, Metal ends up straddling omnivores and univores. Because of this, even in the 1993 GSS data, its association with markers of status was ambiguous, leading Tampubolon to conclude that ``[p]reference or dislike for heavy metal is, therefore, orthogonal to status as measured by education'' \citeyearpar[][257]{tampubolon2008revisiting}, and that ``heavy metal is definitely not a low culture in the sense of culture `strongly associated' or very much liked by those with low education'' (\textit{ibid}). 

Goldberg \citeyearpar{goldberg2011mapping}, using the Relational Class Analysis (RCA) technique, found further evidence of the ambiguous status of Metal using the same GSS 1933 data source. In Goldberg's analysis, whether education was positively, negatively, or not correlated with a preference for Metal depended on the Relational Class that the respondent fell into, leading Goldberg to conclude that ``\ldots [M]etal seems to simultaneously function as a signal of high status according to the Contempo-Trad logic and of low status according to the Hi-Low logic'' \citeyearpar[][1421]{goldberg2011mapping}.

Tampubolon's and Goldberg's argument regarding the ambiguous status of the Metal macrogenre dovetail with the results reported by Lizardo and Skiles \citeyearpar{lizardo_skiles15}, using the same data source as in this paper. Yes, while Metal was still the most disliked genre in 2012, \textit{compared} to 1993, it was one of the macrogenre labels (Rap and Hip Hop being the other) that experienced the steepest \textit{declines} in being disliked in the intervening two decades, especially among members of younger cohorts. This led Lizardo and Skiles \citeyearpar[][18]{lizardo_skiles15} to conclude that ``The fact that young, highly educated Americans are now about as equally unlikely to dislike\ldots Heavy Metal as their same-age non-college-educated peers constitutes a dramatic reversal of the pattern noted by Bryson\ldots [Metal] has experienced an improvement in standing among the college-educated across almost all levels of age.'' This strongly suggests that enjoying some types of Metal music may be an ``emerging form'' of cultural capital---as defined by \citet{prieur2013emerging}---among specific segments of the young elite.  

\subsubsection{Metal Microgenre Analysis}
So, will the real Heavy Metal please stand up? Is Metal a form of emerging cultural capital with potential appeal to young middle-class people or a genre heavily favored by white men with low education? Or is that an outdated picture of the American musical field? Is Metal even correlated with status markers like education at all, as Tampubolon suggested? Is speaking of Metal as a ``youth'' music accurate, or has its audience been aging along with the genre (going on more than four decades and counting)? 

Given prior work, a strong expectation is that, at the very least, we should observe a microgenre partition separating the ``prototypical'' Metal (appealing to low-education white men) and perhaps a less-prototypical one, appealing to younger audiences with high-level cultural capital. Other micro-variations may also show up, possibly tracking generational distinction among members of the metal subculture \citep{koch2020evolutionary}.

\begin{figure}[t!]
    \captionsetup[subfigure]{font=footnotesize}
    \caption{Coefficient estimates from Linear Probability Models predicting macrogenre and microgenre choices for Metal and Salsa using socio-demographic predictors.}
    \label{tab:reg}
    \centering    
     \begin{subfigure}[b]{0.9\textwidth}
        \includegraphics[trim={0 10cm 7cm 0},clip, width=1.0\textwidth]{Tabs/reg-tab-metal.png}
        % trim={<left> <lower> <right> <upper>}
        %\includegraphics[width=1.0\textwidth]{}
    \end{subfigure} \\
     \begin{subfigure}[b]{0.9\textwidth}
        \includegraphics[trim={0 10cm 7cm 0},clip, width=1.0\textwidth]{Tabs/reg-tab-salsa.png}
    \end{subfigure}
\end{figure}
    
To address these questions, I partitioned the Heavy-Metal branch of the dendrogram at a height yielding three microgenre variations (see Figure~\ref{fig:dend-micro-metal}). To examine the issue of audience segmentation, I specify a series of linear probability\footnote{Logit models return substantively identical results, so I stick with LPMs due to ease of interpretation and computation.} regression models with four outcomes: First, a binary variable indicating that a person chose (liked and listened to) the standard macrogenre label (Heavy Metal), and then three models with the microgenre variations as the outcome. 

The results are shown in Table~\ref{tab:reg}.\footnote{As an anonymous reader smartly pointed out, another way of interpreting the results of the link-clustering procedure is splitting the outcome variable into multiple versions, the correlates of which then provide evidence of multiple, but substantively distinct ``causal recipes'' to similar outcomes. This equifinality principle is a crucial interpretative and analytic precept in the Qualitative Comparative Analysis (QCA) tradition, one which has systematic linkages to two-mode network analysis cases by variable data \citep{breiger2014comparative}.} The main predictors are the respondent's education, an indicator variable for whether at least one of the respondent's parents has a college degree, the respondent's age, racial identification (in five categories with ``white'' as the reference), and gender identification (in two categories with ``man'' as the reference).\footnote{Measures to values:
    \begin{itemize}
        \item[--] Education.- Ordinal variable: (1) Less than high school, (2) High school, (3) Associate's Degree, (4) Some college, (5) Bachelor's degree, (6) M.A. degree, (7) Doctoral or Professional degree.
        \item[--] Age.- Ordinal variable: (1) 18--19, (2) 20--25, (3) 25--29, (4) 30--34, (5) 35--39, (6), 40--44, (7), 45--49, (8), 50--54, (9) 55--59, (10), 60--64, (11) 65--69, (12) 70--74, (13) 75--79, (14) 80+.
    \end{itemize}}
Because the age effect is specified using a squared polynomial, it can be hard to interpret its substantive significance. To help with this, I show the predicted probabilities obtained for the model at different age values for the four different microgenres in Figure~\ref{fig:age-metal}.

\begin{figure}[t!]
     \begin{subfigure}[b]{0.45\textwidth}
        \centering    
        \includegraphics[width=1.0\textwidth]{Plots/Micro/micro-by-age-metal.png}
        \caption{}
        \label{fig:age-metal}
    \end{subfigure}
     \begin{subfigure}[b]{0.45\textwidth}
        \centering    
        \includegraphics[width=1.0\textwidth]{Plots/Micro/micro-by-age-salsa.png}
        \caption{}
        \label{fig:age-salsa}
    \end{subfigure}
\end{figure}

As shown in the first model, the socio-demographic predictors of choosing the ``Metal'' macrogenre label are the ones we would expect from Bryson's original work and the stereotypical perception of the genre's audience held by Americans. Metal fans are less likely to have high levels of education, are less likely to be older, are more likely to be white (negative effects for all other racial identifications compared to the base category), and are more likely to be men ($p<0.01$). So much for the emerging cultural capital story?

Not so fast. The results differ when looking at the predictors of other microgenre choices. Indeed, the predictors of choosing the first microgenre variation, shown in the second column of the table, are consistent with Tampubolon's story. For this microgenre, there is a flat education gradient ($p = 0.54$), and importantly, it is more likely to be consumed by people whose parents have a college degree ($p < 0.05$), indicating a link to cultural capital in the home environment \citep{bourdieu84}. This microgenre is tilted towards younger audiences, as only the linear negative age term is statistically significant (see Figure~\ref{fig:age-metal}) and is also neutral concerning race and gender ($p > 0.05$). Given that this microgenre version of Metal appeals mainly to young middle-class respondents and does not repel women or non-white audiences, it starkly contrasts what we would have concluded from observing the macrogenre label effects. 

Note that the other two microgenre variations, whose respective coefficient estimates are shown in the third and fourth columns of the table, conform more closely to the audience segmentation we would expect from the version of Metal. These two microgenres mainly appeal to white men of relatively low education. The main difference between these two microgenres is generational location, with the second microgenre primarily appealing to young adults and middle-aged men and the third variant attracting a younger crowd (see Figure~\ref{fig:age-metal}). This means that the two microgenre variations that most closely conform to the usual audience expectations represent partitions \textit{within} the ``univore'' consumption profile, indicative of distinctive versions of Metal with an appeal to different generational niches \citep{koch2020evolutionary}. The link-clustering approach can tell the difference between these Metal variations, separated by a generational gap (see Figure~\ref{fig:age-metal}).

\subsection{Salsa}
The Salsa music genre has not had as storied a career in the sociology of taste as Metal. It has been under-researched and, for the most part, ignored. I include it in the following analysis due to a shining exception to this pattern, represented by a qualitative study by \citet{Bachmayer2014-pk}, which strongly suggests internal microgenre differentiation within this broad macrogenre label relevant to the reception and consumption side. This differentiation is keyed to hierarchical distinctions across Salsa microgenres based on social status and the concomitant internalization of class-based aesthetic dispositions \citep{lizardo2012reconceptualizing}. 

To examine microgenre variation within Salsa, Bachmayer et al. elicited a list of artists by asking experts to freely name performers and bands and then classify the music typical of the named acts into an ordinal rank based on their judgments of aesthetic quality (``superior,'', ``middlebrow,'' ``inferior''). They then selected the eight artists mentioned by at least three of the four expert informants who displayed a high degree of concordance across experts in terms of their aesthetic rank. Eight compositions experts judged typical of each artist's output were used in guided interviews with forty first-generation Latinx immigrants in the Netherlands and Switzerland (there were no relevant cross-country differences). The most highly regarded expert pieces belonged to the sub-genre ``Classic Salsa''---essentially early salsa performers that set the standard in the 1960s, 1970s, and 1980s---the middlebrow pieces were mainly variants of ``romantic salsa.'' The less aesthetically worthy pieces generally belonged to more recent, radio-friendly styles of ``pop salsa.'' Note that this establishes a strict alignment between microgenre aesthetic standing and the historical sequencing of microgenre emergence within the salsa musical field. We should thus expect this to be reflected in the generational location of each microgenre audience.

What did \citet{Bachmayer2014-pk} find? The short answer is that they found a microcosm of a Bourdieusian world---with some wrinkles. More educated respondents gravitated towards classic salsa and explicitly denigrated audiences that went for more popular styles---enacting a version of Ien Ang's \citeyearpar{ang2000dallas} ``ideology of mass culture'' in this field. High-status respondents emphasized compositional and thematic complexity over ``fun'' aspects like rhythm and ``danceability,'' elements crucial in working-class Latinx immigrants' explanations for why they gravitated towards more romantic and pop styles (finding classic salsa sleepy and boring). Importantly, ``expert consumers'' (people who worked in the musical field in some capacity) demonstrated the same aestheticized appreciation for classic salsa and rejection of popular styles regardless of class background. \citep[][62]{Bachmayer2014-pk} conclude that ``[t]aste in salsa music shows a strong internal, hierarchical order between `artistic' versus `popular' styles that \ldots structures the taste patterns of different social classes.'' 

This means that, at a minimum, we should expect the Salsa microgenres elicited by the link clustering procedure to reflect this hierarchical divide, with a microgenre more likely to be preferred by high-status respondents, particularly the higher-educated---the status marker that was most crucial in Bachmayer et al.'s study. 

\subsubsection{Salsa Microgenre Analysis}
Table~\ref{tab:reg} shows the result of a series of regression analyses analogous to those I specified for the three Metal microgenres, but this time using binary variables indexing choices of the three Salsa microgenres as outcomes (see Figure~\ref{fig:dend-micro-salsa}). The first column shows a model predicting choosing the ``Latin/Spanish/Salsa" macrogenre label using the usual socio-demographic covariates. The following three columns show the predictors of choosing each microgenre variation extracted from cutting the macrogenre label branch of the dendrogram at the required height. Looking at the first column, we would have concluded that choosing Salsa is not predicted by either respondent's ($p = 0.26$) or parent's ($p = 0.17$) education, suggesting that taste for this genre is orthogonal to cultural capital and status. Instead, Salsa is mainly a genre preferred by young respondents (negative linear effect of age) who identify as Hispanic (no surprise there) and as women ($p < 0.05$). 

Like Metal, the microgenre analysis tells a different story. Looking at the third column of the table, predicting choice for the second of the induced microgenres, we find that, indeed, there is a Salsa microgenre statistically more likely to be preferred by highly educated people whose parents are also more likely to be highly educated ($p < 0.05$). Moreover, the age effect on the choice of this microgenre is \textit{reversed} compared to baseline expectations from the macrogenre regression. This microgenre is more likely to be preferred by older respondents (see Figure~\ref{fig:age-salsa}). This microgenre contrasts strongly with the first one extracted by the link clustering procedure, which is overwhelmingly more likely to be preferred by younger respondents, and the third one, which has a relatively flat age gradient. 

These results point to two conclusions: First, there is hierarchical differentiation among microgenres extracted from the ``Latin/Spanish/Salsa'' macrogenre, with one microgenre standing out by its exclusive appeal to high-cultural-capital respondents. Second, this hierarchical differentiation is keyed to age, consistent with the idea that the most prestigious Salsa microgenre is also the ``oldest'' and most traditional \citep{Bachmayer2014-pk}. The other two Salsa microgenres are neutral concerning education ($p > 0.05$) but differ from the high-cultural capital variant regarding generational location. The link-clustering procedure is sensitive to these distinctions, thus allowing us to differentiate between these micro-variations.

\section{Discussion and Conclusion}
Let us take stock. We began by discussing the network or ``relational'' revolution in the quantitative study of taste in sociology \citep{pachucki2010cultural, vlegels2017music, lizardo14}. We noted that many good things have come from considering survey data on taste using a relational lens, including a deeper specification of core concepts in the literature and even discovering new phenomena and empirical patterns. However, we also noted that survey data are as good as the labels chosen to collect the data. 

A resurgent line of critics question whether the labels that appear in our most venerable survey-based studies are true ``genres'' in the sociological (or even stylistic) sense \citep{lena2015relational, vlegels2015music}. The labels are broad, likely to be interpreted by people in heterogeneous ways, and thus hide as much as they reveal. I noted recent attempts to drop the idea of vague macrogenre labels and either study actual sociological genres on the ground (thus partially abandoning studies of audience segmentation or at least radically reconfigure them) or ``dropping the label'' \citep{sonnett2016ambivalence} by querying people about more focused objects of taste (e.g., performers within genres). These are all essential and good developments, but we also noted that they may throw away a good thing. Upping the relationality by exploiting hidden patterns in the same old vague macrogenre data collected before can reveal focused microgenres. 

I proposed to do this using recent developments in the discovery of \textit{overlapping communities} in networks \citep{ahn_etal10}, which partially answers two of the challenges of macrogenre critics: The fact that actual genres are overlapping and not crisply bounded, and the fact that there is hidden heterogeneity within the broad labels we usually focus on. As revealed by the two macrogenre ``case studies,'' focusing on microgenres shows variations and patterns of cultural choice (as well as audience segmentation) that we would not have noticed using the standard approach (in this, the critics are correct). Moreover, these variations capture, at a large-scale level, intuitions and suspicions people have harbored in previous work regarding those particular macrogenres levels while revealing novel insights. For instance, we can be pretty sure that there is indeed a version of Metal that is copacetic with the tastes of certain high-cultural capital youth cutting across gender and racial boundaries \citep{tampubolon2008revisiting}, even if most versions of Metal continue to exist within its traditional socio-demographic niche of less-educated white men \citep{bryson96}. In the same way, we can detect a clear status partition in the field of Salsa music consumption, with an older elite crowd separated by a generational divide from a younger audience set, very likely keyed the distinction between the aesthetically consecrated ``classics'' and more recent popular fare \citep{Bachmayer2014-pk}. More importantly, for the themes we began with, we could discern these patterns of micro-variation of this \textit{without} dropping the label or querying people about hundreds of micro-styles or performers (most of which they would be unfamiliar with). 

Instead, we exploited the venerable principle, etched into classical approaches to defining genres in network terms \citep{dimaggio1987classification}, noting that if genres are defined by the people who choose them, then they are also defined by the other choices that people make when they choose them; the \textit{Country} people combine with one version of \textit{Classical} may not be the same \textit{Country} that combines with a version of the \textit{Blues} (and neither is the same as the \textit{Country} that refuses to combine with any other style). Reciprocally, neither is the \textit{Classical} that combines with \textit{Country} the same as others. Accordingly, the approach proposed here is ripe to be applied to all ``old'' data sources and cultural participation, potentially extracting and revealing many unexploited insights. 

Note that although I have been using the labels ``macro'' and ``micro'' as if they referred to substantive or objective partitions, they are best interpreted as {\em relative} to a given classification level. Thus, microgenres are ``micro'' relative to the usual (perhaps ``basic'' in Rosch's \citeyearpar{Rosch1978-ue} sense) categorization level of vague macrogenre labels populating most arts participation surveys. At any given classification level, further microgenres will be nested within any given micro-partition. In this way, microgenre communities are bound to reflect substantively valid ways people partition and understand the cultural world, whereby there is always the possibility of making finer-grained distinctions within fine-grained distinctions. 

Note, however, that the distinctions uncovered by the link-clustering approach based on the way audiences combine genres or not, may or may not necessarily correspond to micro-\textit{musiciological} distinctions (e.g., ``old school rap," ``1980s old school rap," ``early 1980s old school rap," and the like).\footnote{I thank an anonymous reader for making this observation.} These are, instead, micro-distinctions within genres \textit{sociologically} defined \citep{lena2015relational}. Now, it could be the case---as strongly hinted in the case of Salsa---that these micro-boundaries do coincide, although whether they do so should be the subject of future work taking a mixed-methods strategy. 

Nevertheless, the approach proposed here is general. It can be applied to studying other genre complexes beyond musical taste (allowing for economic data collection) and other processes beyond taste. This includes belief, opinion, and attitude data. Essentially, it enables us to move from ``vague'' responses to more focused responses by exploiting the hidden patterns in the inter-response network formed when people respond to other items. The same lesson learned in studying genres in the sociology of taste applies to the study of opinion and belief data more generally \citep{boutyline2017belief}. To the extent that these are applied to items that are ``vaguely'' defined, they will yield even vaguer (and perhaps less than helpful) meta-classifications of those beliefs and attitudes (like ``liberal'' and ``conservative''). These are the primary ways we interpret how people split themselves into groups by the beliefs they choose to hold, but they are decreasingly apt today if they ever were. 
 

\bibliography{refs.bib}
\end{document}
