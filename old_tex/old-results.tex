Ultimately, there is no magical ``right'' answer here. Still, for analytic purposes, $c = 3, k = 102, max(N_{mg}) = X, min(N_{mg}) = Y$ is sufficiently fine-grained to showcase the analytic advantages of the micro-genre approach. The resulting size distribution is shown in Figure~\ref{fig:size-dist}. The main panel shows the micro-genre communities chosen by 75 or more people, and the inset shows the entire size distribution. micro-genre communities are named according to the original macro-genre label, followed by a number. As the figure shows, the largest micro-genre communities in the data pertain to micro-variants of Indie/Alternative Rock, Contemporary Rock, Reggae, Classic Rock, Blues/R\&B, and Gospel. 


\subsubsection{Hidden Dimensionality}
So what do we learn about people's tastes and about the micro-genres themselves after subjecting our data to link clustering? Comparing the answers we get from applying a standard, relatively undemanding (in terms of statistical assumptions foisted upon the data), namely, Principal Components Analysis, to the original and link-clustered cultural network data (hereafter OD, and LCD, respectively) can be illuminating in this respect. Particularly in terms of showing what is revealed by the link clustering in exposing the hidden relationality in the data and in what ways relying on vague macro-genre levels leads us astray both substantively and theoretically.

\begin{figure}[ht!]
    \includegraphics[width=1.0\textwidth]{Figs/Link Clust/macro-v-micro-pca-eig.png}
    \caption{}
    \label{fig:eig}
 \end{figure}

The first---perhaps already evident---thing we learn is that macrogenre labels hide underlying multidimensionality in cultural taste data, which the link clustering reveals. To show this, Figure \ref{fig:eig} shows the scree plot of the PCA analysis of the OD and the LCD. In the figure, the size of the eigenvalue is shown on the y-axis; the order (first, second, third, and so forth) of the corresponding eigenvalue is shown on the x-axis. The vertical red line going across the plot corresponds to the value on the y-axis where $\lambda = 1$. The black vertical line separates the eigenvalues for which $\lambda >= 1$ (on the left side of the plot) from those for which $\lambda < 1$ (on the right side of the plot). Using the rule of thumb that the number of eigenvalues for which $\lambda > 1$ point to significant dimensions of variation in the data, the PCA on the OD suggests the relevance of perhaps four dimensions that we may want to consider (usually, most analysts stick to the first two). The corresponding scree plot for the LCD reveals upwards of {\em twenty-five} dimensions that could be relevant. 

The first lesson we learn is that {\em macro-genre labels hide relevant dimensions of genre-level (and thus potential person-level) of differentiation in the data}. As we will see next, the artificial lower-dimensionality of the data considered by relying on macro-genre levels leads to substantively misleading answers to our questions.  

\subsubsection{Misleading Answers to Substantive Questions}
 What does the dimension-reduction analysis reveal substantively? Figures \ref{fig:macro-pca} and \ref{fig:micro-pca} show the result of ``classifying'' the genres using the first four dimensions of the PCA on the OD and the first twenty-five dimensions of the PCA on the LCD. In the figures, genres are classified using a ``hybrid'' (hierarchical/k-means) clustering of the (Euclidean) genre distance matrix in four-dimensional space (for the OD) and twenty-five-dimensional space (for the LCD). The number of clusters corresponds to the number of dimensions in the space; four macro-genre clusters ($k_{MG}$) and twenty-five micro-genre clusters ($k_{mg}$). After clustering, macro and micro-genres are re-embedded into two-dimensional space using their respective scores along the first two Principal Coordinates. 
 
\begin{figure}[ht!]
     \begin{subfigure}[b]{1.0\textwidth}
        \centering
        \includegraphics[width=0.6\textwidth]{Figs/Link Clust/macro-pca-clust.png}
        \caption{}
        \label{fig:macro-pca}
    \end{subfigure} 
     \begin{subfigure}[b]{1.0\textwidth}
        \centering
        \includegraphics[width=0.6\textwidth]{Figs/Link Clust/micro-pca-clust.png}
        \caption{}
        \label{fig:micro-pca}
    \end{subfigure} 
    \caption{}
    \label{fig:macro-v-micro-pca}
 \end{figure}
 
 Figure~\ref{fig:macro-pca} suggests that a two-dimensional space aptly represents distinctions between macro-genres (accounting for about 65\% of the variance). Moreover, the macro-genre clusters correspond to the usual logics or discourses that populate many studies in the sociology of taste. On the upper-left, we have a variety of ``folk'' genres, including Country, Gospel, Bluegrass, and the eponymous Folk (with ``Easy'' as a boundary macrogenre). On the lower-left, we have the usual set of ``highbrow'' genres (with Jazz as a boundary macrogenre). On the right, we have the standard set of ``Pop/Industry'' macrogenres, perhaps partitioned, in the four-cluster solution, into ``Rock/Pop" (the varieties of Rock and Roll, with ``Classic Rock/Oldies as a boundary macrogenre) and ``Afro/Pop" (Rap, Reggae, Blues) variants (with ``Latin/Spanish" as a boundary macrogenre). 
 
 Suppose we follow the now well-established ``structuralist'' approach to interpreting these types of embeddings of cultural objects as a space governed by binary oppositions popularized by Bourdieu \citeyearpar{bourdieu84}. In that case, we might say that the first dimension separates or opposes Highbrow and Folk discourses to Pop/Industry logics. The second dimension opposes the (perhaps high-status) highbrow discourse to the perhaps lower-status folk discourse. The fact that Pop/Industry genres are less differentiated on this second, seemingly status-focused dimension might indicate less relevance of status distinctions among these macro-genres. This is all consistent with the usual story of the persistence of these broad meta-macro-genre distinctions and oppositions over time (e.g., in the case of the U.S.) or across national settings. 

\begin{figure}[ht!]
    \includegraphics[width=1.0\textwidth]{Figs/Link Clust/macro-v-micro-clust.png}
    \caption{}
    \label{fig:macro-v-micro-cluster}
 \end{figure}
 
 Two contrasts are evident in the corresponding PCA-cluster plot (Figure~\ref{fig:micro-pca} for the LCD. First, the two-dimensional embedding represents the data poorly (covering less than 10\% of the variance). This makes sense since the LCD is a higher-dimensional entity to begin with. Of more substantive importance, note that the cluster solution presents a more complicated picture that is harder to fit into the stylized story of opposition and differentiation between broad discourses or logics. In Figure~\ref{fig:micro-pca}, micro-genres that belong to opposed or differentiated macro-discourses in Figure~\ref{fig:macro-pca} mingle freely into ``micro-discourses'' or ``micro-logics'' (e.g., combining variants of Country, Classic Rock, Classic, Gospel, and Pop) that would be harder to characterize except as varieties of ``omnivorousness.'' The problem with this approach, however, is that we will end up with more varieties of omnivorousness that we will know what to do with. 

 Figure~\ref{fig:macro-v-micro-pca} makes this point evident. The twenty-five micro-genre clusters shown in Figure~\ref{fig:micro-pca} are plotted along the y-axis. The x-axis consists of the unordered four-category variable formed by the (labeled) clusters in Figure~\ref{fig:macro-pca}. Thus, micro-genres in the same horizontal ``row'' belong to the same micro-genre cluster; reading down the columns assigns each micro-genre to a macro-genre discourse based on their macro-genre ``parent'' label. The point of Figure~\ref{fig:macro-v-micro-pca} is evident; {\em micro-genres combine in ways that typically violate the system of oppositions and differentiation posed by macro-genre discourses}. In that respect, it is clear that the way people combine micro-genres does not follow the rule book suggested by the usual dimension-reduction analyses that rely on macro-genre labels like those shown in Figure~\ref{fig:macro-pca}. Such an approach could not make sense of the micro-genre cluster combining Classical, Opera, Gospel, Country, and Classic Rock ($k_{mg}=11)$ or the set of micro-genre clusters on the bottom left of the figure combining various (presumed) ``highbrow'' and ``Folk'' micro-genres. 
 
 Of course, the standard approach cannot explain these micro-genre combination patterns. Some micro-genre clusters (e.g., $12 >=k_{mg}<=16$) stay within designated macro-genre discourse boundaries, consisting mainly of Rock/Pop and Afro/Pop combinations. Towards the top right, we see that the (very) vague macro-genre ``Classic Rock'' splits into micro-genre variants with the affinity to combine with various ``Country/Folk'' and "Highbrow'' micro-genres. The point of the micro-genre critique, however, is that it is very likely that the {\em type} of Classic Rock that plays well with Rap is different, stylistically and in terms of underlying audience, from the one that plays well with Country. In the same way, the type of Rap listened to by people who also like Classic Rock is likely to differ in consequential ways from the type of Rap listened to by people who also like to listen to Country. 

 Overall, we learned our second lesson: Audiences combine micro-genres in ways that do not respect the traditional boundaries suggested by standard data-reduction techniques applied to macro-genre labels. As such, some substantive conclusions, such as those regarding the existence of logics or discourses that explain affinities and oppositions between macro-genre categories prevalent in previous work, are likely to have been overstated and, in most cases, misleading. The principles of vision, division, and classification people use to combine and divide micro-genre categories seem to be keyed to distinctions that cut across macro-genre categories. 

 \subsubsection{Audience Segmentation and Microgenre Heterogeneity}
 A key task in the sociology of taste is linking audience engagement with particular cultural objects and genre categories to specific sociodemographic markers, primarily of class position, education, and socio-economic status, but also of such identity categories as gender, age (or generation), region of residence, and the like. This approach is crucial in getting a sense of the location of genres in social space and in making inferences about the ``status'' of genre categories themselves. 

\begin{figure}[ht!]
    \centering
    \includegraphics[width=0.8\textwidth]{Figs/Link Clust/classic-rock-macro-demog.png}
    \caption{}
    \label{fig:classic-rock-main}
\end{figure}
 
For instance, we can ascertain whether a given cluster of genres is indeed ``highbrow'' if such high-status markers of social position like higher education diplomas, high income, managerial/professional occupational status, and the like are correlated with their consumption (and the same for ``lower status'' genre categories \citep{bryson96}. The microgenre critique suggests that this crucial task will be controverted if macrogenre labels do not correspond to phenomenologically valid objects from the perspective of personal choices. The conclusions we may reach about the ``status'' of particular genre categories will be misleading if the macrogenre labels hide substantively important and relevant audience heterogeneity at the microgenre level.

\begin{figure}[ht!]
    \centering
    \includegraphics[width=1.0\textwidth]{Figs/Link Clust/classic-rock-fav.png}
    \caption{}
    \label{fig:fav}
\end{figure}

Let us take the case of the (very) vague (and hybrid) macrogenre label ``Classic Rock/Oldies'' as a case study (a similar exercise could be done with each of the twenty macrogenre labels). This is an instructive example since the link clustering approach divides this very broad macrogenre into the most microgenres (twelve). As shown in Figure~\ref{fig:macro-v-micro-cluster}, these micro-genres combine with most of the other micro-genres, indicative of substantial levels of stylistic diversity (and cultural meaning) within the category. If we were to follow the traditional approach, an inquiry as to the audience segmentation pattern of the vague macrogenre label ``Classic Rock,'' by, for instance, specifying a logistic regression with a binary indicator that equals one if the person reports both liking and listening to the genre and a variety of socio-demographic predictors on the right-hand side, we would end up with the results shown in Figure ~\ref{fig:classic-rock-main}. 

\begin{figure}[ht!]
    \centering
    \includegraphics[width=1.0\textwidth]{Figs/Link Clust/classic-rock-age.png}
    \caption{}
    \label{fig:age}
\end{figure}
 
In the Figure, different socio-demographic characteristics are on the y-axis, and the probability of selecting the macrogenre is on the x-axis. The points represented the predicted probability, obtained from the logistic regression model, that a person with that characteristic reports listening to and liking ``Classic Rock''; the horizontal line around the point represents the 90\% confidence interval around the prediction. The red vertical line sits at the point on the x-axis, indicating the base-probability (sample average) of engaging ``Classic Rock''; points to the right of the red line indicating socio-demographic segments that are overrepresented among Classic Rock engagers, and points to the left, represent social categories underrepresented among ``Classic Rock'' engagers. 

\begin{figure}[ht!]
    \centering
    \includegraphics[width=1.0\textwidth]{Figs/Link Clust/classic-rock-educ.png}
    \caption{}
    \label{fig:educ}
\end{figure}

If we proceed in the usual fashion, we would draw various substantive conclusions regarding ``Classic Rock'' from these results. For instance, we would conclude, looking at the curvilinear---``inverted u-shape''---pattern for the effect of age, that ``Classic Rock'' is decidedly ``middle-aged'' genre, being engaged primarily by people in their 40s, 50s, and 60s--peaking for those in their late 50s---but being less likely to be engaged young adults (people younger than 40) or the old (people in their late seventies and eighties). We would also conclude that ``Classic Rock'' is a decidedly ``white'' genre, with white people being overrepresented among listeners, Black and Asian people being underrepresented, and Hispanic people being no more or less likely to engage it. We would also conclude that, beyond the higher likelihood of people without a college degree engaging ``Classic Rock," the genre does not have much class, regional, or gender differentiation (with men, women, people of class designations, and regions comparably likely to engage it). In other words, we would conclude that in the U.S. as of the early 2010s, ``Classic Rock'' is a generationally declining, raced (as white), partially classed (concerning education) musical genre that otherwise functions as a socio-demographically ``unmarked'' form of popular culture (it is the most chosen genre in the survey). 

The corresponding analysis for the twelve microgenre variations, shown in Figures~\ref{fig:age} through~\ref{fig:race} shows that pretty much every single one of these conclusions is either wrong, misleading, or both. Each figure presents the twelve ``Classic Rock'' variations in a separate panel, with the corresponding predictor categories on the y-axis and the (scaled) predicted probability on the x-axis. To facilitate comparison, the shape of the effect (for age, education, and class) for all variations is presented in light gray, and the corresponding effect for that ``Classic Rock'' microgenre is highlighted. Figures~\ref{fig:gender} and \ref{fig:race} show a dot plot using the same data visualization strategy. Additionally, to get more insight regarding the ``Classic Rock'' taste communities revealed by the microgenre analysis, Figure~\ref{fig:fav} plots the probability (on the x-axis) that people who chose each of the ``Classic Rock'' microgenres (on separate plot panels) reports choosing one of the twenty original macrogenres as their favorite. To facilitate interpretation, only macrogenres that are one-half a standard deviation above the predicted value are highlighted, indicating the strongest macrogenre preferences.

\begin{figure}[ht!]
    \centering
    \includegraphics[width=1.0\textwidth]{Figs/Link Clust/classic-rock-class.png}
    \caption{}
    \label{fig:class}
\end{figure}
 
First, take the characterization of ``Classic Rock'' as a generationally ``middle-aged'' genre. As shown in Figure~\ref{fig:age}, the shape of the age effect varies consequentially across the different microgenre variations. As such, we could conclude that ``Classic Rock'' is relatively age-neutral (1), tilts young (2, 4), is preferred by people in their forties (3, 12), or tilts towards senior citizens (5 and 6). Even for those ``Classic Rock'' microgenres for which the peak of the inverted u-shaped effect for age is preserved for people in their fifties and sixties, the peak varies systematically. Accordingly, the characterization of ``middle-aged'' would range from people in their 50s (10, 11) or their sixties (7, 8, 9). 

\begin{figure}[ht!]
    \centering
    \includegraphics[width=1.0\textwidth]{Figs/Link Clust/classic-rock-gender.png}
    \caption{}
    \label{fig:gender}
\end{figure}

Looking at Figures~\ref{fig:educ} and Figure~\ref{fig:class}, we would also have to revise our conclusions about the alleged class neutrality of ``Classic Rock'' as a form of popular culture. While, indeed, some variants are class-neutral, others variants are decidedly class-inflected. For instance, (1) repels people without educational qualifications but attracts high-income people; (3) on the other hand, repels high-income people but attracts the highly educated. Variation (8) has an extreme tilt toward the less educated, and repels upper middle class and high-income people; as Figure~\ref{fig:fav} shows, fans of this Classic Rock variation are likely to report considering Bluegrass and Country as their favorite macrogenres. Other microgenre variations of ``Classic Rock'' like (10) and (11), appeal primarily to a socio-economically upper-middle class and high-income audience. Note that as shown in Figure~\ref{fig:fav} this is a variant combining preferentially with microgenre variations of Folk, Easy, and Country. These people are likely to report considering ``Classic Rock'' and Easy Listening as their favorite macrogenres. 

A similar story emerges for gender. While Figure~\ref{fig:classic-rock-main}, suggests that the macrogenre label ``Classic Rock'' is only weakly gendered, Figure~\ref{fig:gender} shows wide variation in the extent to which different microgenre variations appeal to men and women. Some like (3), (5), and (7) tilt toward women; others like (4), (11), and, particularly (12) tilt toward men. Note that fans of microgenre (12) are also disproportionately likely to report having Hard Rock/Heavy Metal as their favorite macrogenre, shedding light on its extreme tilt toward men. 

\begin{figure}[ht!]
    \centering
    \includegraphics[width=1.0\textwidth]{Figs/Link Clust/classic-rock-race.png}
    \caption{}
    \label{fig:race}
\end{figure}

Looking at Figure~\ref{fig:race}, we can conclude that when it comes to ``Classic Rock'' having a racial tilt toward white people, the macrogenre analysis is not too misleading; most of the predicted probability estimates for this ethnoracial category are to the right of the expected mean. The same goes for its lack of appeal for people who racially identify as Black; most of the predicted probability estimates for this ethnoracial category are to the left of the expected mean for this socio-demographic category. However, the macrogenre analysis hides a lot of variation in the appeal of different variants of ``Classic Rock'' for both Hispanics and Asians. People whos self-identify as Hispanic are attracted to (1), (2), (3), and particularly (10) and (12). People who identify as Asian, on the other hand, are attracted to (5) and (6); note that as shown in Figure~\ref{fig:fav}, fans of these last two microgenre variations are also likely to report having such macrogenres as Classical, Broadway Showtunes, and Jazz as their favorites, as noted earlier these two variants also tend to have the oldest audiences. 
