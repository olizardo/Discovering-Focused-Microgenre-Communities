\documentclass[a4paper,12pt]{extarticle}
\usepackage[utf8]{inputenc}
\usepackage{geometry}
    \geometry{a4paper}
\title{Highlights}

\begin{document}
\maketitle
\begin{itemize}
    \item Recent work has begun to use relational approaches to study taste
    \item However, this work relies on vague genre labels
    \item These labels hide macrogenres, overstating boundaries
    \item I propose a link-clustering approach
    \item The proposed approach uncovers valid microgenres
\end{itemize}
\end{document}

Recent work in the sociology of taste has begun to grapple with the relational properties of traditional survey-based data using techniques inspired by network analysis. Despite productive results from this endeavor, critics rightly question the face and ecological validity of the vague macrogenre labels included in standard arts participation surveys (e.g., Classical, Rock, Rap), which feed into these novel methods. In this paper, I propose a link-clustering approach for discovering focused microgenres from standard survey-based information on cultural tastes, exploiting the underlying relational patterns realized by the indirect connectivity structure of genres (via people) in a two-mode network. The link-clustering approach partially answers two of the challenges of macrogenre critics: The fact that actual genres are overlapping and not crisply bounded and the fact that there is hidden heterogeneity within the broad labels we usually focus on. The results reveal an intuitive partition of traditional macrogenres into focused microgenres, capable of separating audiences along relevant sociodemographic markers, such as education, gender and ethnoracial identities, and age/generation. 
